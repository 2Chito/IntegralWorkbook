%! TEX root = ../main.tex
\documentclass[main]{subfiles}
\usetikzlibrary{shapes.geometric}

\begin{document}
\begin{titlepage}
    % --- 背景の幾何学模様 (重なる円) ---
    \begin{tikzpicture}[remember picture, overlay]
        % ページ全体を座標として使う設定
        \begin{scope}[shift={(current page.south west)}, x={(current page.north east)}, y={(current page.north west)}]
            
            % 50個の円をランダムな位置・大きさ・色で描画
            \foreach \i in {1,...,50} {
                % 乱数を設定
                \pgfmathsetmacro{\randx}{rnd} % x座標 (0.0 - 1.0)
                \pgfmathsetmacro{\randy}{rnd} % y座標 (0.0 - 1.0)
                \pgfmathsetmacro{\randr}{1.5 + 3*rnd} % 半径 (1.5 - 4.5)
                \pgfmathsetmacro{\randc}{20 + 40*rnd} % 色の濃さ (20 - 60)
                \pgfmathsetmacro{\rando}{0.1 + 0.3*rnd} % 透明度 (0.1 - 0.4)

                % 円を描画
                \fill[cyan!\randc!blue, opacity=\rando] (\randx, \randy) circle (\randr cm);
            }
        \end{scope}
    \end{tikzpicture}

    % --- 中央のコンテンツ ---
    % 前面のテキストが見やすいように、半透明の白い長方形を挟む
    \centering
    \begin{tikzpicture}
         \node[fill=white, fill opacity=0.8, text opacity=1, inner sep=1cm, minimum width=15cm, minimum height=22cm, rounded corners=5pt] {
            \begin{minipage}{0.8\textwidth}
                \centering
                \vspace*{2cm}
                {\fontsize{32pt}{40pt}\selectfont \bfseries 今週の積分 \par}
                \vspace{1cm}
                {\fontsize{16pt}{20pt}\selectfont - from ヨビノリたくみ - \par}
                \vspace{6cm}
                {\fontsize{12pt}{18pt}\selectfont 作成者: 藤川 二千翔 \par}
                \vspace{0.5cm}
                {\fontsize{12pt}{18pt}\selectfont 作成日: \today \par}
                \vspace*{1cm}
            \end{minipage}
        };
    \end{tikzpicture}
\end{titlepage}

\end{document}