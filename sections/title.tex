%! TEX root = ../main.tex
\documentclass[main]{subfiles}
\usetikzlibrary{shapes.geometric}

\begin{document}
\begin{titlepage}
    % --- 色の定義 ---
    % 背景を黒色に変更
    \definecolor{FGBg}{RGB}{0,0,0}
    % ロゴ用の金色
    \definecolor{FGgold}{RGB}{212,175,55}

    % ページの背景色と文字色を設定
    \pagecolor{FGBg}
    \color{white}

    % ページ全体のコンテンツを中央揃えに
    \centering

    % 上部の余白
    \vspace*{1cm}

    % --- 金色の細い線 ---
    \begin{tikzpicture}
        \draw [FGgold, line width=0.8pt] (0,0) -- (8cm, 0);
    \end{tikzpicture}

    \vspace*{1cm}

    % --- "Focus Gold" ロゴ ---
    % フォントはTimes系で代用
    {\fontfamily{qtm}\selectfont
        {\fontsize{50}{60}\selectfont \color{FGgold} Focus Gold}
    }

    \vspace*{1cm}

% --- 「今週の積分」のタイトル ---
% フォントは原ノ味フォント(別途設定が必要な場合あり)
{\fontsize{50}{60}\selectfont \bfseries 今週の積分}

    \vspace*{5mm}

    % --- 金色の細い線 ---
    \begin{tikzpicture}
        \draw [FGgold, line width=0.8pt] (0,0) -- (8cm, 0);
    \end{tikzpicture}

% --- 下部の余白を確保するための空白 ---
\vfill

% --- 出版社名 ---
{\Large 啓林館}

    % 下部の余白
    \vspace*{1cm}

\end{titlepage}

% 次のページから背景色をリセット
\nopagecolor

\end{document}